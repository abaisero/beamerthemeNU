\documentclass[9pt]{beamer}

\usepackage{minted}

\usetheme{NU}
\title{Example Document Using the NU Beamer Theme}
\subtitle{Flashy Subtitle}

\author[A. Baisero]{Andrea Baisero\\\texttt{baisero.a@northeastern.edu}}
\venue[Venue 2020]{Venue 2020, Online}
\institute{Northeastern University, Boston, USA}

\addtitlepagelogo{logo.png}
\addtitlepagelogo{another_logo.png}

\begin{document}

\section{Overview}

% Use
%
%     \begin{frame}[allowframebreaks]
%
% if the TOC does not fit one frame.
\begin{frame}
  %
  \frametitle{Ordinary table of contents}

  \tableofcontents

\end{frame}

\begin{frame}
  %
  \frametitle{Table of contents that highlights current section}

  \tableofcontents[currentsection]

\end{frame}

\section{Features}

\begin{frame}
  %
  \frametitle{Table of contents that highlights current section}

  \tableofcontents[currentsection]

\end{frame}

\hidelogo

\begin{frame}
  %
  \frametitle{Package Options}

  Options can be used to change some theme properties, e.g.,
  %
  \mint{latex}|\usetheme[t,SansSerif,AMS]{NU}|

  \begin{block}{Vertical Alignment}
    %
    \begin{description}
      %
      \item[c] text is aligned at the center of the frame. (default)
      %
      \item[t] text is aligned at the top of the frame.
      %
    \end{description}
    %
  \end{block}

  \begin{block}{Fonts}
    %
    \begin{description}
      %
      \item[MathSerif]  Only math is formatted in serif. (default)
      %
      \item[SansSerif]  Nothing is formatted in serif.
      %
      \item[Serif]  All text is formatted in serif.
      %
    \end{description}
    %
  \end{block}

  \begin{block}{Enumeration Style of Theorems}
    %
    \begin{description}
      %
      \item[unnumbered]  Theorems are not numbered. (default)
      %
      \item[numbered]  Theorems are numbered.
      %
      \item[AMS]  AMS style.
      %
    \end{description}
    %
  \end{block}

\end{frame}

\showlogo

\begin{frame}
  %
  \frametitle{Highlighting \& Blocks}

  Text can be highlighted:
  %
  \begin{itemize}
    %
    \item With command \structure{structure}, which gives text the NU color.
    %
    \item With command \localstructure{localstructure}, which gives local color.
    %
    \item With command \alert{alert}, which gives text extra umph.
    %
  \end{itemize}

  \begin{block}{Block}
    %
    This is a standard block, and its associated \localstructure{local color}.
    %
  \end{block}

  \begin{example}
    %
    This is an ``example''  block, and its associated \localstructure{local color}.
    %
  \end{example}

  \begin{alertblock}{Alertblock}
    %
    This is an ``alert''  block, and its associated \localstructure{local color}.
    %
  \end{alertblock}

\end{frame}

\hidelogo

\begin{frame}
  %
  \frametitle{Lists}

  \begin{itemize}
    %
    \item Bullet lists are marked with a red dot.
    %
  \end{itemize}

  \begin{enumerate}
    %
    \item Numbered lists are marked with a white number in a red circle.
    %
  \end{enumerate}

  \begin{description}
    %
    \item[Description] highlights important words with red text.
    %
  \end{description}

  \begin{example}

    \begin{itemize}
      %
      \item Lists change color in block environments.
      %
    \end{itemize}

    \begin{enumerate}
      %
      \item As seen in this example block.
      %
    \end{enumerate}

    \begin{description}
      %
      \item[Description] and here
      %
    \end{description}

  \end{example}

  \begin{alertblock}{Another Example}

    \begin{itemize}
      %
      \item Lists change color in block environments.
      %
    \end{itemize}

    \begin{enumerate}
      %
      \item As seen in this alert block.
      %
    \end{enumerate}

    \begin{description}
      %
      \item[Description] and here
      %
    \end{description}

  \end{alertblock}

\end{frame}

\showlogo

\begin{frame}
  %
  \frametitle{Venue Command}

  This package introduces the \structure{venue} command, with which you can
  describe the venue where you are giving the talk, e.g. conference name,
  location, etc.\newline

  The command is used like \structure{author}, \structure{institute}, etc.,
  %
  \mint{latex}|\venue [Venue 2020]{Venue 2020, Online}|

\end{frame}

\begin{frame}[fragile]
  %
  \frametitle{Title Page}

  \begin{block}{Title Page Logos}

    Logos can be added to the title page using
    %
    \begin{minted}{latex}
\addtitlepagelogo{logo.png}
\addtitlepagelogo{other_logo.png}
    \end{minted}

    Calling the command multiple times will add multiple logos.
    %
  \end{block}

  \begin{block}{Hide Frame Logo}

    If necessary, it is possible to free up space in individual frames by
    omitting the logo on the bottom right.  Hiding the logo (and making it
    reappear) is possible by using the following commands between frames:

    \mint{latex}|\hidelogo|

    \mint{latex}|\showlogo|
    %
  \end{block}

\end{frame}

\hidelogo

\begin{frame}
  %
  \frametitle{Theorem-like Environments}

  The theme defines additional theorem-like environments:

  \vspace{2ex}

  \begin{itemize}
    %
    \begin{minipage}{.45\linewidth}
      %
      \item Conjecture
      %
      \item Facts
      %
      \item Hypothesis
      %
      \item Observation
      %
      \item Proposition
      %
      \item Assumption
      %
      \item Axiom
      %
    \end{minipage}
    %
    \begin{minipage}{.45\linewidth}
      %
      \item Property
      %
      \item Calculation
      %
      \item Computation
      %
      \item Notation
      %
      \item Remark
      %
      \item Remarks
      %
    \end{minipage}
    %
  \end{itemize}

  \begin{notation}[Necessary Notation]
    %
    \ldots
    %
  \end{notation}

  \begin{axiom}[Arbitrary Axiom]
    %
    \ldots
    %
  \end{axiom}

  \begin{property}[Peculiar Property]
    %
    \ldots
    %
  \end{property}

\end{frame}

\showlogo

\section{References}

\begin{frame}
  %
  \frametitle{Table of contents that highlights current section}

  \tableofcontents[currentsection]

\end{frame}

\hidelogo

\begin{frame}[allowframebreaks]
  %
  \frametitle{References}

  \begin{thebibliography}{}

      % Article is the default.
      \setbeamertemplate{bibliography item}[book]

      \bibitem{Hartshorne1977}
      R.~Hartshorne.
      \newblock \emph{Algebraic Geometry}.
      \newblock Springer-Verlag, 1977.

      \setbeamertemplate{bibliography item}[article]

      \bibitem{Artin1966}
      M.~Artin.
      \newblock On isolated rational singularities of surfaces.
      \newblock \emph{Amer. J. Math.}, 80(1):129--136, 1966.

      \setbeamertemplate{bibliography item}[online]

      \bibitem{Vakil2006}
      R.~Vakil.
      \newblock \emph{The moduli space of curves and Gromov--Witten theory}, 2006.
      \newblock \url{http://arxiv.org/abs/math/0602347}

      \setbeamertemplate{bibliography item}[triangle]

      \bibitem{AM1969}
      M.~Atiyah og I.~Macdonald.
      \newblock \emph{Introduction to commutative algebra}.
      \newblock Addison-Wesley Publishing Co., Reading, Mass.-London-Don
      Mills, Ont., 1969

      \setbeamertemplate{bibliography item}[text]

      \bibitem{Fraleigh1967}
      J.~Fraleigh.
      \newblock \emph{A first course in abstract algebra}.
      \newblock Addison-Wesley Publishing Co., Reading, Mass.-London-Don Mills, Ont., 1967

  \end{thebibliography}
  %
\end{frame}

\end{document}
